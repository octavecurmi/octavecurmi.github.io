%%%%%%%%%%%%%%%%%%%%%%%%%%%%%%%%%%%%%%%%%
% "ModernCV" CV and Cover Letter
% LaTeX Template
% Version 1.3 (29/10/16)
%
% This template has been downloaded from:
% http://www.LaTeXTemplates.com
%
% Original author:
% Xavier Danaux (xdanaux@gmail.com) with modifications by:
% Vel (vel@latextemplates.com)
%
% License:
% CC BY-NC-SA 3.0 (http://creativecommons.org/licenses/by-nc-sa/3.0/)
%
% Important note:
% This template requires the moderncv.cls and .sty files to be in the same 
% directory as this .tex file. These files provide the resume style and themes 
% used for structuring the document.
%
%%%%%%%%%%%%%%%%%%%%%%%%%%%%%%%%%%%%%%%%%

%----------------------------------------------------------------------------------------
%	PACKAGES AND OTHER DOCUMENT CONFIGURATIONS
%----------------------------------------------------------------------------------------

\documentclass[11pt,a4paper,sans]{moderncv} % Font sizes: 10, 11, or 12; paper sizes: a4paper, letterpaper, a5paper, legalpaper, executivepaper or landscape; font families: sans or roman

\moderncvstyle{casual} % CV theme - options include: 'casual' (default), 'classic', 'oldstyle' and 'banking'
\moderncvcolor{blue} % CV color - options include: 'blue' (default), 'orange', 'green', 'red', 'purple', 'grey' and 'black'

\usepackage{lipsum} % Used for inserting dummy 'Lorem ipsum' text into the template

\usepackage[scale=0.75]{geometry} % Reduce document margins
%\setlength{\hintscolumnwidth}{3cm} % Uncomment to change the width of the dates column
%\setlength{\makecvtitlenamewidth}{10cm} % For the 'classic' style, uncomment to adjust the width of the space allocated to your name
\usepackage[french]{babel}
\usepackage[utf8]{inputenc}
%----------------------------------------------------------------------------------------
%	NAME AND CONTACT INFORMATION SECTION
%----------------------------------------------------------------------------------------

\firstname{Octave} % Your first name
\familyname{Curmi} % Your last name
% All information in this block is optional, comment out any lines you don't need
\title{
\newline
Curriculum vitae
\newline
Avril 2021}
%\address{49 rue Corneille}{59000 Lille France}
%\mobile{+33 628434324}
%\email{octave.curmi@ens-rennes.fr}
%\extrainfo{October 21, 1992}
%----------------------------------------------------------------------------------------

\begin{document}

%----------------------------------------------------------------------------------------
%	COVER LETTER
%----------------------------------------------------------------------------------------

% To remove the cover letter, comment out this entire block

%\clearpage

%\recipient{HR Department}{Corporation\\123 Pleasant Lane\\12345 City, State} % Letter recipient
%\date{\today} % Letter date
%\opening{Dear Sir or Madam,} % Opening greeting
%\closing{Sincerely yours,} % Closing phrase
%\enclosure[Attached]{curriculum vit\ae{}} % List of enclosed documents

%\makelettertitle % Print letter title

%\lipsum[1-2] % Dummy text
%\lipsum[4] % Dummy text

%\makeletterclosing % Print letter signature


%----------------------------------------------------------------------------------------
%	CURRICULUM VITAE
%----------------------------------------------------------------------------------------

\makecvtitle % Print the CV title

%----------------------------------------------------------------------------------------
%	EDUCATION SECTION
%----------------------------------------------------------------------------------------
\section{A propos}
\cventry{}{curmi@renyi.hu}{}{}{}{}
\cventry{}{Né le 21 Octobre 1992}{}{}{}{}

\section{Études}

\cventry{Oct 2015 -- Juin 2019}{Doctorant en Mathématiques}{"Topologie des singularités non-isolées de surfaces", sous la direction de Patrick Popescu-Pampu}{Université de Lille, France}{}{Soutenue le 17 Juin 2019. Jury compososé de:
\hfill
\begin{itemize}
\item \textbf{Javier Fern\'andez de Bobadilla}, Rapporteur.
\item \textbf{Hussein Mourtada}.
\item \textbf{Anne Pichon}, Rapporteur.
\item \textbf{Martin Saralegui Aranguren}.
\item \textbf{Bernard Teissier}.
\end{itemize}
}  % Arguments not required can be left empty
\cventry{2014 -- 2015}{Master en Mathématiques fondamentales}{}{Université de Lille}{}{}
\cventry{2014}{Agrégé de mathématiques}{}{}{}{}

%\section{Mémoire de Master}
%
%\cvitem{Titre}{\emph{Le théorème de Bernstein-Kushnirenko-Khovanskii}}
%\cvitem{Directeur}{Patrick Popescu-Pampu, Université Lille 1}
%\cvitem{Description}{Ce mémoire explorait différentes preuves du théorème de Bernstein-Kushnirenko-Khovanskii, qui exprime le nombre de points d'une intersection complète générique de dimension zéro dans un tore algébrique comme un volume mixte, et quelques généralisations de ce résultat.}

%----------------------------------------------------------------------------------------
%	WORK EXPERIENCE SECTION
%----------------------------------------------------------------------------------------

\section{Expérience}
\subsection{Publications}
\cventry{}{Topology of smoothings of non-isolated singularities of complex surfaces}{}{published in  Math. Ann. 377, 1711–1755 (2020)}{disponible à https://arxiv.org/pdf/1910.04145.pdf}{}

\cventry{}{Boundary of the Milnor fiber of a Newton non-degenerate surface singularity}{Published in  Adv. Math. 372 (2020), Article ID 107281, 31 pages}{Disponible à \url{https://arxiv.org/pdf/1911.13258.pdf}}{}{}

\subsection{Prépublications}
\cventry{Aug 2020}{A proof of A. Gabrielov's rank theorem}{61 pages}{with A. Belotto da Silva and G. Rond}{Disponible à \url{https://arxiv.org/pdf/2008.13130.pdf}}{}

\subsection{Postes}
\cventry{Sept 2020 -}{Postdoc}{}{Institut Alfré Rényi, Budapest, Hongrie}{}{}
\cventry{Sept 2018 - Août 2020}{ATER}{}{Université d'Aix-Marseille}{}{}

\subsection{Exposés en conférences sur invitation}
\cventry{Nov. 2020}{``A new proof of Gabrielov's rank Theorem''}{GDR singularities and applications, online conference}{}{}{}{}{}
\cventry{Nov 2019}{``Topology of smoothings of non-isolated singularities of complex surfaces''}{Workshop on Topological and Analytical methods in singularity theory, Guanajuato, Mexique}{}{}{}{}
\cventry{Juil 2019}{``Topology of smoothings of non-isolated singularities of complex surfaces''}{Session de singularités de la conférence ``Joint meeting France-Brazil in Mathematics'', IMPA, Rio, Brésil}{}{}{}{}
\cventry{Oct 2018}{``Topology of non-isolated singularities of complex surfaces''}{Workshop ``Lipschitz Geometry of Singularities''}{Casa Matem\'atica Oaxaca, Oaxaca, Mexique}{}{}
\subsection{Minicours et groupes de travail}
\cventry{Fév 2021-}{J'anime le groupe de travail ``around toric geometry and topology of singularities'', qui consiste en un exposé hebdomadaire et vise à offrir une compréhension approfondie de l'algorithme de calcul de bord de fibre de Milnor que je propose dans le contexte torique, et des questions attenantes.}{}{Institut Alfréd Rényi, Budapest}{}{}{}
\cventry{Juil 2019}{Minicours autour de l'article de Mumford ``The topology of normal singularities of an algebraic surface and a criterion for simplicity''}{}{Galatasaray University, Istanbul, Turquie}{}{}{} 
\cventry{Mai 2019}{Minicours ``Around Milnor's fibration theorem: an interplay of algebra and topology''}{1st Najah short courses in mathematics}{An-Najah University, Naplouse, Palestine}{}{}{} 
\cventry{Fév 2019}{Quelques bases de géométrie torique, I \& II}{Groupe de travail de singularités}{Université d'Aix-Marseille, France}{}{}
\subsection{Exposés en séminaires}
\cventry{Mars 2021}{``Boundary of the Milnor fiber of a Newton non-degenerate non-isolated singularity of surface''}{Webinaire de singularités d'Aix-Marseille}{}{}{}
\cventry{Mars 2021}{``A proof of Gabrielov's rank Theorem''}{ Iberoamerican webinar of Young researchers in Singularity Theory}{}{}{}
\cventry{Fév. 2021}{``A proof of Gabrielov's rank Theorem''}{ Séminaire de Géométrie Analytique}{Université de Rennes}{}{}
\cventry{Déc. 2020}{``A proof of Gabrielov's rank Theorem''}{ Séminaire commun Géométrie Algébrique - Angers, Géométrie, Topologie, Algèbre - Nantes}{}{}{}
\cventry{Oct. 2020}{``A proof of Gabrielov's rank Theorem''}{ Séminaire Algèbre et Géométrie}{UNAM, Cuernavaca, Mexique}{}{}
\cventry{Juin 2019}{``Topologie des lissages de singularités non-isolées de surfaces complexes''}{Séminaire hebdomadaire de topologie, géométrie et algèbre}{Université de Nantes, France}{}{}
\cventry{Mai 2019}{``Topologie des lissages de singularités non-isolées de surfaces complexes''}{Séminaire hebdomadaire de singularités}{Université Paris Diderot, France}{}{}
\cventry{Avr 2019}{``Topologie des lissages de singularités non-isolées de surfaces complexes''}{Séminaire hebdomadaire de géométrie algébrique}{Université de Lille, France}{}{}
\cventry{Avr 2019}{``Topology of smoothings of non-isolated singularities of complex surfaces''}{Séminaire hebdomadaire Singacom}{Universidad de Valladolid, Espagne}{}{}
\cventry{Mar 2019}{``Topologie des lissages de singularités non-isolées de surfaces complexes''}{Séminaire hebdomadaire de systèmes dynamiques et géométrie}{Université d'Angers, France}{}{}
\cventry{Nov 2018}{``Topologie des lissages de singularités non-isolées de surfaces complexes''}{Séminaire hebdomadaire de géométrie complexe}{Université d'Aix-Marseille, France}{}{}

\vspace{0.1cm}

\cventry{}{Tous les exposés qui suivent furent donnés dans le cadre du séminaire "géométrie des espaces singuliers" à l'Université de Lille}{}{}{}{}

\vspace{0.1cm}

\cventry{Nov 2017}{\textit{"Autour de la formule de Klein pour les courbes planes"}}{}{}{}{}
\cventry{Nov 2017}{\textit{"Les formules de Plücker pour les courbes planes"}}{}{}{}{}
\cventry{Avr 2017}{\textit{"Les singularités non-isolées de surfaces étudiées à la Némethi-Szilard"}}{}{}{}{}
\cventry{Jan 2017}{\textit{"Topologie des singularités non-dégénérées normalisées de surfaces"}}{}{}{}{}
\cventry{Juin 2016}{\textit{"Topologie des singularités isolées non-dégénérées de surfaces, d'après Oka"}}{}{}{}{}
\cventry{Fév 2016}{\textit{"Points entiers dans les polygones et genre des courbes dans le tore"}}{}{}{}{}

\subsection{Exposés de vulgarisation}
\cventry{Juil 2019}{``Singularities in Topology''}{Exposé devant une large audience d'étudiants et enseignants}{Lebanese International University, Beyrouth, Liban}{}{}{}


\subsection{Conférences auxquelles j'ai assisté}
\cventry{Nov 2019}{``Workshop ``Topological and analytical methods in singularity theory''}{}{CIMAT, Guanajuato, Mexique}{}{}{}
\cventry{Juil 2019}{``Joint meeting France-Brazil in Mathematics''}{}{IMPA, Rio, Brésil}{}{}{}{}
\cventry{Mai 2019}{``Geometry and Topology of Singularities''}{}{Alfréd Rényi Institute of Mathematics, Budapest, Hongrie}{}{}{}{}
\cventry{Oct 2018}{Workshop ``Lipschitz Geometry of Singularities''}{}{Casa Matem\'atica Oaxaca, Oaxaca, Mexique}{}{}
\cventry{Juin 2018}{International school on Singularities and Lipschitz Geometry}{}{Unidad Cuernavaca del Instituto de Matem\'aticas, UNAM, Mexique}{}{}
\cventry{Déc 2017}{Theory of Valuations}{}{Université Paris 7}{}{}
\cventry{Juil 2017}{Galois Meets Newton: Algebraic and Geometric aspects of Singularity Theory}{Conference in honor of A. Khovanskii}{Weizmann Institute of Science, Israel}{}{}
\cventry{Fév 2017}{A panorama on singular varieties}{Conference in honor of Lê Dũng Tráng}{Institut de Mathématiques de l'Université de Séville IMUS, Espagne}{}{}
\cventry{Nov 2016}{The 4th Franco-Japanese-Vietnamese Singularities}{}{Université Savoie-Mont Blanc, Chambéry}{}{}
\cventry{Oct 2016}{Resolution of foliations}{}{Université Paris 7}{}{}
\cventry{Mai 2016}{Nonisolated singularities}{}{Université de Lille}{}{}
\cventry{Mai 2016}{Algebra, geometry and topology of singularities}{Une semaine d'école et une semaine de de conférences}{Université Galatasaray, Istanbul, Turquie}{}{}
\cventry{Fév 2016}{Winter Braids VI}{School on braids and low dimensional topology}{Université de Lille}{}{} 

\subsection{Participation à des projets de recherche}
\cventry{2021}{Membre du projet ``Local Properties of Subanalytic Sets'', avec A. Belotto da Silva et G. Rond d'Aix-MArseille Université}{Soutenu par le programme ``international emerging actions'' du CNRS}{}{}{}

\cventry{2019}{Impliqué dans le projet ``Jets, arcs and invariants of surface singularities''}{avec M. Tosun (Université Galatasaray, Turquie), C. Plénat (Aix-Marseille université, France), H. Mourtada (Université Paris-Diderot, France), M. Ruggiero (Université Paris-Diderot, France) and B. Karadeniz (Université Galatasaray, Turquie)}{Projet de recherche Franco-Turc, impliquant un séjour à Istanbul en Juillet 2019}{}{}


\subsection{Enseignement}
\cventry{2019 -- 2020}{Divers modules de Mathématiques en L1, L2 et L3}{176 heures}{Université d'Aix-Marseille, France}{}{}{}
\cventry{2018 -- 2019}{Cours d'analyse et d'algèbre en première et deuxième année de classe préparatoire intégrée}{192 heures}{Ecole Polytech Marseille, Marseille, France}{}{}{}
\cventry{2015 -- 2018}{Chargé de TD à l'Institut Mines-Télécom}{64h par an}{Université de Lille}{}{}{}
\cventry{Sept 2015}{Professeur remplaçant}{}{collège Desrousseaux d'Armentières (59)}{}{}{}
\cventry{2014-2015}{Khôlleur en première année de classe préparatoire aux écoles de commerce}{}{Lycée Saint Paul, Lille}{}{}
\cventry{2011 - 2017}{Professeur particulier}{}{Dans tous types de contextes.}{}{}
%----------------------------------------------------------------------------------------
%	COMPUTER SKILLS SECTION
%----------------------------------------------------------------------------------------

\section{Informatique}

\cvitem{Programmation}{\textsc{CAML}, C++, Singular}
\cvitem{Logiciel}{\LaTeX, OpenOffice, Linux, Microsoft Windows, Maple, Geogebra, IPE}

%----------------------------------------------------------------------------------------
%	LANGUAGES SECTION
%----------------------------------------------------------------------------------------

\section{Langues}

\cvitemwithcomment{Français}{Langue maternelle}{}
\cvitemwithcomment{Anglais}{Courant}{}
\cvitemwithcomment{All.}{Basique}{Vocabulaire de base}
\cvitemwithcomment{Arabe, Esp.}{Basique}{En train d'apprendre}
%----------------------------------------------------------------------------------------
%	INTERESTS SECTION
%----------------------------------------------------------------------------------------

%\section{Centres d'intérêts}
%
%\cvitem{Hobbies}{Sport, échecs, lecture, musique, théâtre,  cuisine}
%\cvitem{Culturel} {histoire, géographie, sciences politiques, voyages}
%\cvitem{Engagements}{Action sociale, transmission du savoir}
%----------------------------------------------------------------------------------------

\end{document}